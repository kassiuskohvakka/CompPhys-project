\documentclass[a4paper,12pt]{article}
\frenchspacing
\usepackage[utf8]{inputenc}
\usepackage{a4wide}
%\usepackage[finnish]{babel}
\usepackage[english]{babel}
\usepackage{mathtools}
\usepackage{siunitx}
%\usepackage[utf8]{inputenc}
\usepackage[pdftex]{graphicx}
\usepackage{icomma}
\usepackage{hyperref}
\usepackage{amssymb}
\usepackage{amsmath}
\usepackage{float}
%\usepackage{mhchem}
\usepackage{parskip}
\usepackage{graphicx}
\usepackage{caption}
\usepackage{subcaption}
\usepackage{fancyhdr}
\usepackage{eurosym}
\usepackage{enumerate}
%\usepackage{subfig}
%\usepackage{floatrow}
%\floatsetup[figure]{style=plain,subcapbesideposition=top}



%%% NEW COMMANDS

\newcommand{\dd}{\,\mathrm{d}}
\newcommand{\exerline}{
\vspace*{.1cm}
\noindent \rule{\textwidth}{1pt}
\vspace*{.1cm}
}

%\usepackage{lastpage}

\pagestyle{fancy}
\fancyhead{}
\fancyhead[LO,RE]{Comp. Phys. -- Project}
%\fancyhead[RO,LE]{Harjoitus 1, 14.9.2012}
\fancyhead[RO,LE]{Kohvakka, 2019}
%\fancyfoot{}
%\fancyfoot[LO,RE]{Kangaslampi / Laaksonen}
%\fancyfoot[RO,LE]{\thepage/\pageref{LastPage}}


\hyphenation{every-where}
\renewcommand{\baselinestretch}{1}

\begin{document}


%\begin{minipage}[t][1.5cm][b]{.2\textwidth}
%\AaltoLogoRandomLarge{0.7}
%\end{minipage}
\begin{minipage}[t][1.5cm][b]{\textwidth}
\begin{center}
\Large{\textbf{Computational Physics}} \\
\vspace*{.1cm}
\Large{\textbf{Project -- 2D FEM Schrödinger solver}}\\
\vspace*{.1cm}
\large{Kassius Kohvakka, 586977}
\end{center}
\end{minipage} 
\vspace{-0.4cm}

\exerline

\section{Background}

\section{Theory and methods}

The two-dimensional time-independent Schrödinger equation is given by

\begin{equation}
\label{eq: 2DSchrodinger}
-\frac{\hbar}{2m} \left( \frac{\partial^2 \psi (x,y)}{\partial x^2} + \frac{\partial^2 \psi (x,y)}{\partial y^2} \right) + V(x,y)\psi (x,y) = E \psi (x,y) .
\end{equation}

In order to use FEM to numerically solve Eq. (\ref{eq: 2DSchrodinger}), we expand the wave function $\psi (x,y)$ in a basis of tetrahedral hat functions $\lbrace \phi_i \rbrace_{i=1..N}$ as

\begin{equation}
\label{eq: basisExpansion}
\psi (x,y) \approx \sum_{i=1}^{N} \alpha_i \phi_i(x,y),
\end{equation}

where $N$ is the number of finite elements used in our computation and $\alpha_i$, the coefficients in the linear combination are to be solved for. The linear combination then allows us to construct an approximate solution to the original problem. Setting, for simplicity, $\frac{\hbar}{m} = 1$, writing the partial derivatives more concisely as $\nabla^2 \psi(x,y)$, and substituting our basis expansion in Eq. (\ref{eq: 2DSchrodinger}), we get 


\begin{equation}
\label{eq: schrodingerInFEMBasis}
\left( -\frac{1}{2} \nabla^2 + V(x,y) \right) \sum_{i=1}^{N} \alpha_i \phi_i(x,y)  = E \sum_{i=1}^{N} \alpha_i \phi_i(x,y).
\end{equation}

Rearranging and multiplying both sides by the basis function $\phi_j$, we acquire

\begin{equation}
\sum_{i=1}^{N} \left[ -\frac{1}{2} \phi_j(x,y) \nabla^2 \phi_i(x,y) + \phi_j(x,y) V(x,y) \phi_i(x,y) \right]  \alpha_i   = E \sum_{i=1}^{N} \alpha_i \phi_j(x,y) \phi_i(x,y).
\end{equation}

We can now integrate both sides over the domain $\Omega$ of our problem (and lighten the notation by getting rid of the cluttering $(x, y)$-silliness) to get

\begin{equation}
\label{eq: almostReadyFEMeq}
\sum_{i=1}^{N} \left[ -\frac{1}{2} \left( \int_{\Omega} \phi_j \nabla^2 \phi_i \dd A \right) + \left( \int_{\Omega} \phi_j V \phi_i \dd A \right) \right]  \alpha_i   = \sum_{i=1}^{N} E \left( \int_{\Omega} \phi_j \phi_i \dd A \right) \alpha_i .
\end{equation}

The integrals inside the ordinary parentheses are now matrices. The first of the three still needs to be rewritten by Green's first identity:

\begin{equation}
\int_{\Omega} \phi_j \nabla^2 \phi_i \dd A = \underbrace{\oint_{\partial\Omega} \phi_j (\nabla \phi_i \cdot \hat{n} )\dd l}_{=0} - \int_{\Omega} \nabla \phi_j \cdot \nabla \phi_i \dd A,
\end{equation}

where the vanishing of the indicated term can be achieved in practice by setting either the basis functions $\lbrace\phi_i\rbrace_i$ or the normal-directional derivatives $\lbrace \nabla \phi_i \cdot \hat{n}\rbrace_i$ at the boundary $\partial \Omega$ to 0 by use of Dirichlet or Neumann boundary conditions, respectively. We can then finally identify the matrices in Eq. (\ref{eq: almostReadyFEMeq}) as the kinetic matrix $T_{ji}$, the potential matrix $V_{ji}$ and the overlap matrix $S_{ji}$:

\begin{equation}
\label{eq: readyFEMeq}
\sum_{i=1}^{N} \left[ \underbrace{\frac{1}{2} \left( \int_{\Omega} \nabla \phi_j \cdot \nabla \phi_i \dd A \right)}_{T_{ji}} + \underbrace{\left( \int_{\Omega} \phi_j V \phi_i \dd A \right)}_{V_{ji}} \right]  \alpha_i   = \sum_{i=1}^{N} E \underbrace{\left( \int_{\Omega} \phi_j \phi_i \dd A \right)}_{S_{ji}} \alpha_i .
\end{equation}

Since the summations on both sides of the equation are just the $j^{\text{th}}$ elements of a matrix-vector product, the elementwise equality implies equality of the resultant vectors and we get

\begin{equation}
\label{eq: matrixFormFEMeq}
(T + V)\alpha = ES\alpha,
\end{equation}

which is a generalized eigenvalue problem involving our known matrices. Solving this, we acquire as eigenvectors the coefficient vectors $\alpha$ approximating the true eigenstates as per the linear combination (\ref{eq: basisExpansion}) and the corresponding approximate energies $E$ of the eigenstates as eigenvalues.



%\begin{figure}[h]
%\centering
%\includegraphics[width=0.7\textwidth]{figs/minimum_energy.pdf}
%\caption{The electron density at the separation $\Delta x$ minimizing the total energy of the system with $M=2$, $N=6$. The vertical lines indicate the locations of the two nuclei. The external potential due to the nuclei is shown in arbitrary units with the dashed line.}
%\label{fig: min_energy}
%\end{figure}

%\begin{figure}[h]
%\centering
%\begin{subfigure}[b]{.5\textwidth}
%  \centering
%  \includegraphics[width=\textwidth]{figs/wave_euler_forward.pdf}
%  \caption{Implicit Euler}
%  \label{fig:sub1}
%\end{subfigure}%
%\begin{subfigure}[b]{.5\textwidth}
%  \centering  
%  \includegraphics[width=\textwidth]{figs/wave_cn_forward.pdf}
%  \caption{Crank-Nicolson}
%  \label{fig:sub2}
%\end{subfigure}
%\caption{The results for the two integration schemes for forward time integration of the wave equation from $t_0=0$ to $t_{f}$ = 0.2}
%\label{fig:waveForward}
%\end{figure}



\end{document}
